\newpage
\section{Experimentální část}
\label{chpt:experiment}

\subsection{Materiál}


% Příklad jednoduchého seznamu položek

\subsubsection{Použité chemikálie}

\begin{itemize}
    \item Adenosintrifosfát (ATP) – NEB, USA
    \item Agarosa – Bio-Rad, USA
    \item Imidazol – Sigma, USA
\end{itemize}


% Ukázka tabulky

\begin{table}[H]
\centering
\caption{\textbf{Ukázková tabulka primerů pro PCR}}
\renewcommand{\arraystretch}{1.4} % zvýší mezeru mezi řádky
\begin{adjustbox}{max width=\textwidth} % umožní přizpůsobení šířky
\begin{tabular}{|l|l|}
\hline
\textbf{Mutace} & \textbf{Sekvence (5' $\rightarrow$ 3')} \\
\hline
Y511A & cgacctgcaggcgcgccgTCAGCGGGTGTATGCGTTGGGTGCCTCGACA \\
\hline
N509A & gcaggcgcgccgTCAGCGGGTGTAGTAAGCGGGTGCCTCGAC \\
\hline
\end{tabular}
\end{adjustbox}
\end{table}


\end{itemize}

\subsubsection{Roztoky}
\begin{table}[H]
\centering
\caption{\textbf{Složení pufrů použitých k proteinové izolaci}}
\label{tab:pufry}
\begin{adjustbox}{max width=\textwidth}
\newcolumntype{Y}{>{\centering\arraybackslash}X}
\begin{tabularx}{\textwidth}{|l|Y|Y|Y|Y|}
\hline
\textbf{Složka} & \textbf{Lyzační pufr} & \textbf{Pufr A} & \textbf{Pufr B} & \textbf{Promývací pufr} \\
\hline
Tris (mM) & 50 & 50 & 50 & 50 \\
KCl (mM) & 500 & 500 & 500 & 0 \\
NaCl (mM) & – & – & – & 2000 \\
Glycerol (\%) & 5{,}0 & 5{,}0 & 5{,}0 & 5{,}0 \\
Imidazol (mM) & 10 & 10 & 500 & 10 \\
$\beta$-merkaptoetanol (mM) & 5 & 5 & 5 & 5 \\
PMSF (mM) & 1 & – & – & – \\
Lysozym (mg/ml) & 0{,}1 & – & – & – \\
\hline
\end{tabularx}
\end{adjustbox}
\begin{flushleft}
\textit{Poznámka:} pH všech pufrů bylo upraveno na 8{,}0 (KOH nebo HCl); pufry byly sterilizovány filtrací (0{,}45~$\mu$m). $\beta$-mercaptoethanol, PMSF a lysozym byly přidávány těsně před použitím.
\end{flushleft}
\end{table}




\subsubsection{Metody}